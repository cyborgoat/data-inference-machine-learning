\documentclass[12pt]{article}
\input{structure.tex} % Include the file specifying the document structure and custom commands
\newcommand{\assignmentQuestionName}{Question} % The word to be used as a prefix to question numbers; example alternatives: Problem, Exercise
\newcommand{\assignmentClass}{BIO101} % Course/class
\newcommand{\assignmentTitle}{Assignment\ \#1} % Assignment title or name
\newcommand{\assignmentAuthorName}{Yosuke Hanamura} % Student name

% Optional (comment lines to remove)
\newcommand{\assignmentClassInstructor}{Jones 10:30am} % Intructor name/time/description
\newcommand{\assignmentDueDate}{Monday,\ January\ 24,\ 2019} % Due date

\begin{document}
	
	% --------------------------------------------------------------
	%                         Start here
	% --------------------------------------------------------------
	
	\title{18-785 Data, Interference, and Applied Machine Learning \\ Homework Assignment 5}
	\author{Junxiao Guo \\ Andrew ID: junxiaog}
	
	\maketitle
	
	% --------------------------------------------------------------
	%     You don't have to mess with anything below this line.
	% --------------------------------------------------------------
	%----------------------------------------------------------------------------------------
	%	QUESTION 1
	%----------------------------------------------------------------------------------------
	

	\section{Statistical learning}
	\subsection{1.1 Rule-based approach}
	\questiontext{Four steps to implementing a rule-based approach}
	\answer{
	\begin{enumerate}
	\item A list of rules or rule base, which is a specific type of knowledge base.
	\item An inference engine or semantic reasoner, which infers information or takes action based on the interaction of input and the rule base. The interpreter executes a production system program by performing the following match-resolve-act cycle.
	\item Temporary working memory.
	\item A user interface or other connection to the outside world through which input and output signals are received and sent.
	\end{enumerate}
	}
	\questiontext{Domain knowledge required to establish a rule}
	\answer{
	}
	\subsection{1.2 Over-fitting}
	\questiontext{Explain over-fitting and why it is a problem in statistical learning}\\\\
	\answer{
	\underline{Over-fitting}: In statistics, over-fitting is "the production of an analysis that corresponds too closely or exactly to a particular set of data, and may therefore fail to fit additional data or predict future observations reliably". An over-fitted model is a statistical model that contains more parameters than can be justified by the data.The essence of over-fitting is to have unknowingly extracted some of the residual variation (i.e. the noise) as if that variation represented underlying model structure.}
	\answer{
	\underline{Why Problematic}: Over-fitted model can not obtain truly unbiased sample of population of any data. The over-fitted model is likely to be biased to the sample instead of predicting the parameters for the entire data population.}


	\questiontext{Dealing with small datasets}
	\answer{
	\underline{For small datasets}: If we only have a small datasets of ten data points, we \textbf{should use s simple model} because if we use a complex model, it will be easily get over-fitted because complex model can easily be trained to fit every data-points from the small datasets, which will lose the generality of our model.}
	\subsection{1.3 Commonly used approaches to avoid over-fitting}
	\answer{
	\textbf{Simplifying the model}: make sure that the number of independent parameters in your fit is much smaller than the number of data points you have.  By independent parameters, I mean the number of coefficients in a polynomial or the number of weights and biases in a neural network, not the number of independent variables
	\\\\
	\textbf{Adding Regularizations}: Regularization attempts to reduce the variance of the estimator by simplifying it, something that will increase the bias, in such a way that the expected error decreases
	}

	
	\subsection{1.4 Metrics used to evaluate the performance}
	\answer{
	\textbf{F1 Score}\\
	F1 is an overall measure of a model's accuracy that combines precision and recall, in that weird way that addition and multiplication just mix two ingredients to make a separate dish altogether.
	$${F_1}= {(recall^{-1}+precision^{-1})\over{2}}^{-1}=2\cdot{precision\cdot recall\over{precision+recall}}$$
	For the precision and recall for F1 Score: 
	$$Precision={TruePositive \over {TruePositive + FalsePostiive}}$$
	$$Recall = {TruePositive \over {TruePositive+FalseNegative}}$$
	\\
	\\
	\textbf{R-Squared}\\
	R-squared, also known as the coefficient of determination, is the statistical measurement of the correlation between an investment’s performance and a specific benchmark index. In other words, it shows what degree a stock or portfolio’s performance can be attributed to a benchmark index.
	$$R^2=1-{MSE(model)\over{MSE(baseline)}}$$
	
	MSE(model) = Mean Squared Error of the predictions against the actual values
	$$MSE(model)=\sum_i^N(y_i-\hat{y})^2$$
	
	MSE(baseline) = Mean Squared Error of  mean prediction against the actual values
	$$MSE(model)=\sum_i^N(\bar{y_i}-\hat{y})^2$$}
\answer{
	\textbf{Examples}\\
	\textbullet { Evaluation of named entity recognition and word segmentation: F1 Score}\\
	\textbullet { Representing how a funds movements correlates with a benchmark index: R-Squared}
	}
	
	\subsection{1.5 Why Benchmark}
	\answer{
	1. Benchmark is standard against which you compare the solutions, to get a feel if the solutions are better or worse.\\\\
	2. When the benchmarks are “representative,” they allow engineering effort to be focused on a small but high-value and widely used set of targets. In the best cases, benchmarks initiate a virtuous circle, propelling a cycle of optimization and improved value for all members of a community}
\answer{
	\textbf{Examples for benchmark}\\
	1. 
}
	

	
	\newpage
	\section{Machine Learning}
	\subsection{2.1 What is Machine Learning \& Why Machine Learning}
	\answer{
	\textbf{What is machine learning}\\
	Machine learning (ML) is the scientific study of algorithms and statistical models that computer systems use to perform a specific task without using explicit instructions, relying on patterns and inference instead. It is seen as a subset of artificial intelligence. Machine learning algorithms build a mathematical model based on sample data, known as "training data", in order to make predictions or decisions without being explicitly programmed to perform the task. Machine learning algorithms are used in a wide variety of applications, such as email filtering and computer vision, where it is difficult or infeasible to develop a conventional algorithm for effectively performing the task.}
	\answer{
	\textbf{Evolution over time}\\
	\textbullet Alan Turing Test(1950) :  A machine can actually learn, if when we communicate with it, we cannot distinguish it from another human.\\
	\textbullet ELIZA(1952) : Arthur Samuel (IBM) wrote the first game-playing program, ELIZA, for checkers, to achieve sufficient skill to challenge a world champion.\\
	\textbullet Neural Network(1957): Frank Rosenblatt (Cornell University) invented the perceptron, a very simple linear classifier.\\
	\textbullet Artificial Intelligence(1990): Computer science and statistics combined to provide a data-driven approach to machine learning.\\
	\textbullet 	Big Data(2010): Exponential growth in the volume, velocity and variety of data available for analysis and research.\\
	\textbullet Open Data(2014): Infrastructure, protocols and standards for providing open access to data.
	}
	

	\subsection{2.2 Examples of machine learning techniques}
	\answer{
	\textbullet Decision Tree (Supervised)\\
	\textbullet Linear Regression (Supervised)\\
	\textbullet K-means (Unsupervised)\\
}
	\subsection{2.3 Classification \& Regression}
	\answer{
	The main difference between them is that the output variable in regression is numerical (or continuous) while that for classification is categorical (or discrete)}
	
	\subsection{2.4 Supervised \& Unsupervised learning}
	\answer{
	\textbf{Supervised}: All data is labeled and the algorithms learn to predict the output from the input data. \\
	\textbf{Unsupervised}: All data is unlabeled and the algorithms learn to inherent structure from the input data.
}
	
	\subsection{2.5 Examples}
	\questiontext{Examples of successful applications of machine learning and technique involved}
	\answer{
	\textbullet \textbf{Facial Recognition}: Convoluted Neural Network\\
	\textbullet \textbf{Speech Recognition}: Recurrent Neural Network\\
}	
	
	
	\section{Diabetes data}
	\subsection{3.1 Correlation \& Heat-Map}
	\answer{
	\textbf{Correlation Matrix}
	\begin{center}
		\includegraphics[width=0.5\columnwidth]{/Users/robert/Documents/CMU/19Fall/18785/18785-data-inf-ml/images/hw5_imgs/hw5_p3q2.png} % Example image
	\end{center}
	\textbf{Heat-map of the Matrix}
	\begin{center}
	\includegraphics[width=0.5\columnwidth]{/Users/robert/Documents/CMU/19Fall/18785/18785-data-inf-ml/images/hw5_imgs/hw5_p3q1.png} % Example image
	\end{center}
	}

	\subsection{3.2 Collinearity}
	\answer{
		Collinearity, in statistics, correlation between predictor variables (or independent variables), such that they express a linear relationship in a regression model.
	}

	\subsection{3.3 Multivariate Model}
	\answer{}
	
	\section{Analyzing the Titanic data set}
	\subsection{4.1 Logistic Regression \& Linear Regression}
	\subsection{4.2 Probability of Survival}
	\subsection{4.3 Table of Survival Probability}
	\subsection{4.4 Logistic Regression Model for Survival Rates}
	\subsection{4.5 Model Performance}
	
	
\end{document}